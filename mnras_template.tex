% mnras_template.tex 
%
% LaTeX template for creating an MNRAS paper
%
% v3.3 released April 2024
% (version numbers match those of mnras.cls)
%
% Copyright (C) Royal Astronomical Society 2015
% Authors:
% Keith T. Smith (Royal Astronomical Society)

% Change log
%
% v3.3 April 2024
%   Updated \pubyear to print the current year automatically
% v3.2 July 2023
%	Updated guidance on use of amssymb package
% v3.0 May 2015
%    Renamed to match the new package name
%    Version number matches mnras.cls
%    A few minor tweaks to wording
% v1.0 September 2013
%    Beta testing only - never publicly released
%    First version: a simple (ish) template for creating an MNRAS paper

%%%%%%%%%%%%%%%%%%%%%%%%%%%%%%%%%%%%%%%%%%%%%%%%%%
% Basic setup. Most papers should leave these options alone.
\documentclass[fleqn,usenatbib]{mnras}

% MNRAS is set in Times font. If you don't have this installed (most LaTeX
% installations will be fine) or prefer the old Computer Modern fonts, comment
% out the following line
\usepackage{newtxtext,newtxmath}
% Depending on your LaTeX fonts installation, you might get better results with one of these:
%\usepackage{mathptmx}
%\usepackage{txfonts}

% Use vector fonts, so it zooms properly in on-screen viewing software
% Don't change these lines unless you know what you are doing
\usepackage[T1]{fontenc}
\usepackage{siunitx}    % Adds units
% Allow "Thomas van Noord" and "Simon de Laguarde" and alike to be sorted by "N" and "L" etc. in the bibliography.
% Write the name in the bibliography as "\VAN{Noord}{Van}{van} Noord, Thomas"
\newcommand{\SL}[1]{{\textcolor{Cerulean}{[SL: #1]}}}
\newcommand{\muB}{\mu_{\rm B}}
\newcommand{\ncor}[1]{{\textcolor{black}{#1}}}
\newcommand{\nncor}[1]{{\textcolor{black}{#1}}}
\def\muS{\mu_{\rm S}}
\DeclareRobustCommand{\VAN}[3]{#2}
\let\VANthebibliography\thebibliography
\def\thebibliography{\DeclareRobustCommand{\VAN}[3]{##3}\VANthebibliography}


%%%%% AUTHORS - PLACE YOUR OWN PACKAGES HERE %%%%%

% Only include extra packages if you really need them. Avoid using amssymb if newtxmath is enabled, as these packages can cause conflicts. newtxmatch covers the same math symbols while producing a consistent Times New Roman font. Common packages are:
\usepackage{graphicx}	% Including figure files
\usepackage{amsmath}	% Advanced maths commands

%%%%%%%%%%%%%%%%%%%%%%%%%%%%%%%%%%%%%%%%%%%%%%%%%%

%%%%% AUTHORS - PLACE YOUR OWN COMMANDS HERE %%%%%

% Please keep new commands to a minimum, and use \newcommand not \def to avoid
% overwriting existing commands. Example:
%\newcommand{\pcm}{\,cm$^{-2}$}	% per cm-squared

%%%%%%%%%%%%%%%%%%%%%%%%%%%%%%%%%%%%%%%%%%%%%%%%%%

%%%%%%%%%%%%%%%%%%% TITLE PAGE %%%%%%%%%%%%%%%%%%%

% Title of the paper, and the short title which is used in the headers.
% Keep the title short and informative.
\title[Short title, max. 45 characters]{MNRAS \LaTeXe\ template -- title goes here}

% The list of authors, and the short list which is used in the headers.
% If you need two or more lines of authors, add an extra line using \newauthor
\author[K. T. Smith et al.]{
Keith T. Smith,$^{1}$\thanks{E-mail: publications@ras.ac.uk (KTS)}
A. N. Other,$^{2}$
Third Author$^{2,3}$
and Fourth Author$^{3}$
\\
% List of institutions
$^{1}$Royal Astronomical Society, Burlington House, Piccadilly, London W1J 0BQ, UK\\
$^{2}$Department, Institution, Street Address, City Postal Code, Country\\
$^{3}$Another Department, Different Institution, Street Address, City Postal Code, Country
}

% These dates will be filled out by the publisher
\date{Accepted XXX. Received YYY; in original form ZZZ}

% Prints the current year, for the copyright statements etc. To achieve a fixed year, replace the expression with a number. 
\pubyear{\the\year{}}

% Don't change these lines
\begin{document}
\label{firstpage}
\pagerange{\pageref{firstpage}--\pageref{lastpage}}
\maketitle

% Abstract of the paper
\begin{abstract}
This is a simple template for authors to write new MNRAS papers.
The abstract should briefly describe the aims, methods, and main results of the paper.
It should be a single paragraph not more than 250 words (200 words for Letters).
No references should appear in the abstract.
\end{abstract}

% Select between one and six entries from the list of approved keywords.
% Don't make up new ones.
\begin{keywords}
keyword1 -- keyword2 -- keyword3
\end{keywords}

%%%%%%%%%%%%%%%%%%%%%%%%%%%%%%%%%%%%%%%%%%%%%%%%%%

%%%%%%%%%%%%%%%%% BODY OF PAPER %%%%%%%%%%%%%%%%%%

\section{Introduction}
The study of dark matter and scalar fields around black holes, both supermassive and primordial and in binary systems, has been approached from relativistic analyses, ultralight dark matter solitons, Bose–Einstein condensates, relativistic accretion, axion stars, superfluidity and gravitational wave effects.
Binary neutron star mergers provide a laboratory for probing fundamental physics through their gravitational- wave emission and electromagnetic counterparts. In particular, they may allow us to explore signatures of physics beyond the Standard Model in strong-gravity regimes, such as those of dark matter. \citep{PhysRevD.88.063522} investigates the dynamics of light dark matter modeled as a minimally coupled scalar field surrounding binary neutron star systems, finding that scalar fields form bound clouds with potential effects on gravitational wave signals, though these remain undetectable with current observatories at realistic densities.


Measurements of the dynamical environment of supermassive black holes (SMBHs) are becoming abundant and precise. \citep{Bar_2019} searches for ultralight dark matter solitons using stellar velocity measurements near Sgr A* and Event Horizon Telescope imaging of M87*, setting constraints that exclude solitons predicted by naive extrapolations of the soliton-halo relation for particle masses $2\times10^{-20}$--$8\times10^{-19}$ eV (Sgr A*) and $\lesssim4\times10^{-22}$ eV (M87*), while showing that SMBH dynamical effects can suppress soliton masses by orders of magnitude.
The effect of a supermassive black hole (SMBH) on the density profile of a fuzzy dark matter (FDM) soliton core at the centre of a dark matter halo has been studied in \citep{10.1093/mnras/staa202}, here the Schrödinger-Poisson equations are numerically solved and demonstrates a `squeezing' effect where the black hole decreases the soliton core radius while increasing central density. Applying this analysis to M87 and the Milky Way with observational constraints, the authors constrain the FDM particle mass to exclude the range $10^{-22.12}$--$10^{-22.06}$ eV and show that improved mass measurements and theoretical modeling can extend these constraints further.

The mass-radius relation of self-gravitating Bose-Einstein condensates with an attractive $-1/r$ external potential created by a central mass was determined in \citep{chavanis2019massradiusrelationselfgravitatingboseeinstein} where an analytical Gaussian ansatz approach has been used to study both noninteracting and self-interacting bosons. These results apply to dark matter halos made of self-gravitating Bose-Einstein condensates where a central mass mimics a supermassive black hole, demonstrating how central black holes affect mass-radius relations and maximum masses of axionic halos, with approximate analytical results compared against exact limits.

A general class of axion models, including the QCD and string axion, in which the PQ symmetry is broken before or during inflation has been considered in \citep{PhysRevD.102.023013}, here the authors discuss axion star formation in virialized dark minihalos around primordial black holes through gravitational Bose-Einstein condensation. The authors determine conditions for minihalos to kinetically produce axion stars before galaxy formation, expecting up to $\sim 10^{17}$ (or $\sim 10^9$ for string axions) axion stars within a 100 parsec radius around the Sun.


The relativistic Bondi accretion of dark matter onto a non-spinning black hole, assuming the dominant halo component is a Standard Model gauge-singlet scalar was self-consistently solved in \citep{Feng_2022}. The study constrains the scalar mass ($m\simeq10^{-5}$ eV) and quartic self-coupling ($\lambda\lesssim10^{-19}$) to be compatible with galactic halo properties. In the hydrodynamic limit the authors find a lower bound on the accretion rate, $\dot{M}_{\rm min}=96\pi G^2M^2 m^4/\lambda\hbar^3$; for $M=10^6~{\rm M}_\odot$ this gives $\dot{M}_{\rm min}\simeq1.41\times10^{-9}~{\rm M}_\odot~{\rm yr}^{-1}$, subdominant to baryonic Eddington accretion. The spike density profile $\rho_0(r)$ within the self-gravitating regime is better represented by a piecewise double-power law, with $\rho_0(r)\propto r^{-1.20}$ near the sound horizon, $\rho_0(r)\propto r^{-1.00}$ toward the Bondi radius, and $\rho_0(r)\propto r^{-1.08}$ in between; this contrasts with steeper $\rho_0(r)\propto r^{-1.75}$ profiles for Coulomb-like self-interactions.

The density profile of superfluid dark matter around supermassive black holes at galactic centres was computed in \citep{De_Luca_2023} where it was shown that, depending on the fluid equation of state, the dark matter profile exhibits distinct power-law behaviours that can distinguish superfluid from collisionless dark matter predictions.

For dark matter to be detectable with gravitational waves from binary black holes, it must reach higher than average densities in their vicinity. Light (wave-like) dark matter density between binaries can be significantly enhanced by accretion from the surrounding environment. \citep{PhysRevLett.132.211401} shows that the dephasing effect on the last ten orbits of an equal-mass binary is maximized when the Compton wavelength of the scalar particle is comparable to the orbital separation, $2\pi/\mu\sim d$. The phenomenology differs from channels typically discussed where dynamical friction and radiation of energy/angular momentum drive dephasing; instead, it is dominated by the radial force towards the overdensity between the black holes. Whilst numerical studies are limited to comparable scales, this effect may persist at larger separations and/or particle masses, playing a role in binary merger history.

\citep{Yin_2024} presents novel findings on the parameter space of axion stars—extended objects forming in dense dark matter environments through gravitational condensation—with emphasis on formation within dense minihalos potentially surrounding primordial black holes and in axion miniclusters. The work investigates the relation between the radius and mass of axion stars in dense surroundings, revealing distinct morphological characteristics compared to isolated scenarios. Applications to the bound state between primordial black holes and axion stars, and gravitational microlensing from extended objects, provide observational constraints on ``halo'' axion stars. Constraints on the galactic axion star population fraction are derived from microlensing events in the EROS-2 survey using numerical solutions of the Schrödinger–Poisson equation.

Gravitational wave observations have significantly broadened our capacity to explore fundamental physics beyond the Standard Model. \citep{ybtp-fzwl} investigates resonant interactions between binary bla     ck hole systems and solitons—self-gravitating configurations of ultralight bosonic dark matter—which induce metric perturbations and generate distinct oscillatory patterns in gravitational waves. Upcoming experiments such as the Laser Interferometer Space Antenna could detect these oscillatory patterns, providing evidence for solitons. Because the effect relies solely on gravity with no coupling of the dark sector to Standard Model particles, future gravitational-wave surveys can probe dark matter.

Superradiant instability can form clouds around rotating black holes (BHs) composed of ultralight bosonic fields, such as axions. A BH with such a cloud in a binary system exhibits rich phenomena, with gravitational waves (GWs) from the merger providing a probe of axions. \citep{takahashi2024selfinteractingaxioncloudsrotating} studies the evolution of axion clouds in binaries during inspiral, including axion self-interaction effects. When self-interaction is significant, two types of clouds coexist through mode coupling. The evolution is examined considering dissipation from both self-interaction and tidal interaction; for tidal interactions, indirect emission via resonant and off-resonant transitions is included as second-order perturbation. The signatures of axion self-interaction are imprinted in GW phase modifications, with dynamical instability (bosenova) identified as a possibility during the binary inspiral phase.



Gravitational waves provide crucial insights about black hole environments. \citep{PhysRevD.110.083011} uses numerical relativity simulations to study self-interacting scalar (wave-like) dark matter clouds accreting onto isolated and binary black holes. Repulsive self-interactions smoothen the ``spike'' of isolated black holes and saturate density, while attractive self-interactions enhance growth and produce cuspy profiles but can become unstable and undergo explosions (bosenova-like) reducing local cloud density. The impact of self-interactions on equal-mass black hole mergers is quantified via gravitational-wave dephasing calculations across coupling ranges. Repulsive self-interactions saturate cloud density and reduce dephasing; for attractive interactions, dephasing may be larger, but if these interactions dominate pre-merger, dark matter can undergo bosenova during inspiral, disrupting the cloud and reducing dephasing.

Light scalar particles are well-motivated dark matter candidates arising naturally in many Standard Model extensions. Gravitational interactions near black holes can trigger growth of dense scalar configurations that alter binary dynamics and imprint signatures on gravitational-wave signals. \citep{roy2025scalarfieldsblackhole} develops a semi-analytic waveform model for binaries in scalar environments, validated against numerical relativity simulations, and applies it in a Bayesian analysis of the LIGO–Virgo–KAGRA catalog. The results set physically meaningful upper bounds on scalar environments around compact binaries; when superradiance priors are included, tentative evidence for such an environment is found in GW190728 with $\ln\mathcal{B}_{\rm vac}^{\rm env} \approx 3.5$, corresponding to a light scalar field with mass $\sim 10^{-12}\,\mathrm{eV}$.

\citep{banik2025bosonstarshostingblack} studies self-gravitating condensates (boson stars) formed from scalar ultra-light dark matter (ULDM) hosting central black holes. The equations of hydrostatic equilibrium are numerically solved in the non-relativistic limit, consistently incorporating the black hole's gravitational potential, to obtain all configurations of boson-star–black-hole systems for different boson star masses, interaction types, and black hole masses. An analytic expression for the density profile is proposed and compared with numerical results, showing good agreement for attractive interactions and finite mass-ratio ranges. The inspiral of boson-star–black-hole systems with secondary smaller black holes is examined, with gravitational-wave dephasing quantified due to the ULDM environment. A Fisher matrix analysis reveals regions of the ULDM mass and self-coupling parameter space that future gravitational-wave observatories such as LISA can probe.

Gravitational waves emitted by massive black hole binaries can be affected by various environmental effects. \citep{Chase:2025wwj} studies how gravitational waves from black holes in quasi-circular orbits are affected by an ultra-light, vector-field dark-matter environment minimally coupled to the binary. This environment induces oscillatory gravitational potentials that perturb the binary orbit and imprint on the binary binding energy and emitted gravitational waves. The environmental effect on gravitational-wave phase is computed using the stationary-phase approximation within the post-Newtonian formalism. A Fisher analysis estimates detectability of this effect with a four-year LISA observation, focusing on vector fields with ultra-light masses in the $(10^{-19}, 10^{-16}) \; \rm{eV}$ range. Space-borne interferometers such as LISA could measure or constrain local vector dark-matter environments, provided dark-matter density exceeds roughly $10^{14} \rm{M}_\odot/{\rm{pc}}^3$.

\citep{CalderonBustillo:2022cja} presents the first systematic search for exotic compact mergers in Advanced LIGO and Virgo events, comparing gravitational-wave signals GW190521, GW190426$\_$190642, GW200220$\_$061928, and trigger 200114$\_$020818 (S200114f) to a catalogue of 759 numerical simulations of head-on mergers of horizonless exotic compact objects (Proca stars)—interpreted as self-gravitating lumps of fuzzy dark matter sourced by ultralight bosonic particles. The Proca-star merger hypothesis is strongly rejected for GW190426, weakly rejected for GW200220, and weakly favoured for GW190521 and S200114f. GW190521 and GW200220 yield highly consistent boson masses of $\mu_{\rm B} = 8.69^{+0.61}_{-0.75}\times10^{-13}$ eV and $\mu_{\rm B} = 9.13^{+1.18}_{-1.30}\times10^{-13}$ eV at $90\%$ credible level. A preliminary population study estimates the fraction of Proca-star mergers as $\zeta = 0.27^{+0.43}_{-0.25}$ (or $0.39^{+0.38}_{-0.33}$ including S200114f), with GW190521 maintained as a Proca-star merger candidate.

Scalar fields with masses between $10^{-21}$ and $10^{-11} \rm{eV}/c^2$ exhibit enhanced gravitational interactions with black holes, forming scalar clouds that modify coalescing binary dynamics and produce gravitational waves that provide a new detection channel for light scalar fields. \citep{Cheng:2025wac} simulates black-hole mergers with mass ratios $q=1$ and $q=1/2$ immersed in scalar field overdensities with masses in the range $M\muS\in[0,1.0]$. The numerical methods employ constraint-satisfying initial data solvers based on puncture methods, with improved accuracy (eighth-order finite differences) and reduced initial orbital eccentricity in the open-source software {\textsc{Canuda}}. The impact of scalar mass on gravitational and scalar radiation is investigated, revealing that binaries can undergo delayed or accelerated mergers relative to vacuum. The work highlights the challenges and importance of accurate modelling of black-hole binaries in dark matter environments.

Binary neutron star mergers provide a laboratory for probing fundamental physics through gravitational-wave emission and electromagnetic counterparts, offering exploration of physics beyond the Standard Model in strong-gravity regimes. \citep{Srikanth:2025lic} investigates the dynamics of light dark matter (minimally coupled scalar field) surrounding binary neutron star systems, assessing whether the scalar field remains bound over late inspiral-merger timescales and its impact on observable signatures. The scalar field forms a common cloud around the binary that does not disperse in a range of scenarios. At sufficiently high densities, measurable effects emerge, including binary inspiral dephasing, less compact post-merger remnants, and dynamical ejecta suppression. For astrophysically motivated densities, however, these effects remain small and undetectable with current or next-generation gravitational-wave observatories.

Ultralight (or fuzzy) dark matter (ULDM) is an alternative to cold dark matter, characterized by solitonic cores at collapsed halo centres. These cores increase drag on supermassive black hole (SMBH) binaries, changing merger dynamics and the gravitational-wave background. \citep{Boey:2025qbo} performs detailed simulations of high-mass SMBH binaries within massive halo solitons, finding more rapid decay than previous simulations and semi-analytic approximations. The drag dependence on ULDM particle mass is confirmed; masses greater than $10^{-21}$ eV could potentially alleviate the final parsec problem and suppress gravitational-wave production at lower pulsar-timing-array frequencies.

When two galaxies merge, they often produce a supermassive black hole binary (SMBHB) at their centre. Numerical simulations with stars and cold dark matter show that SMBHBs typically stall at parsec separations for billions of years—the final parsec problem. \citep{Koo:2023gfm} suggests that ultralight dark matter (ULDM) halos around SMBHBs can generate dark matter waves via dynamical friction that carry away orbital energy, rapidly driving black holes together. Numerical simulations of black hole binaries inside ULDM halos demonstrate that gravitational cooling and quasi-normal modes avoid the loss-cone problem. The decay timescale provides lower bounds on ULDM particle and SMBH masses consistent with observational data, demonstrating that ULDM waves drive rapid orbital decay of black hole binaries.



\section{Conclusions}

The last numbered section should briefly summarise what has been done, and describe
the final conclusions which the authors draw from their work.

\section*{Acknowledgements}

The Acknowledgements section is not numbered. Here you can thank helpful
colleagues, acknowledge funding agencies, telescopes and facilities used etc.
Try to keep it short.

%%%%%%%%%%%%%%%%%%%%%%%%%%%%%%%%%%%%%%%%%%%%%%%%%%
\section*{Data Availability}

 
The inclusion of a Data Availability Statement is a requirement for articles published in MNRAS. Data Availability Statements provide a standardised format for readers to understand the availability of data underlying the research results described in the article. The statement may refer to original data generated in the course of the study or to third-party data analysed in the article. The statement should describe and provide means of access, where possible, by linking to the data or providing the required accession numbers for the relevant databases or DOIs.




%%%%%%%%%%%%%%%%%%%% REFERENCES %%%%%%%%%%%%%%%%%%

% The best way to enter references is to use BibTeX:

\bibliographystyle{mnras}
\bibliography{example} % if your bibtex file is called example.bib


% Alternatively you could enter them by hand, like this:
% This method is tedious and prone to error if you have lots of references
%\begin{thebibliography}{99}
%\bibitem[\protect\citeauthoryear{Author}{2012}]{Author2012}
%Author A.~N., 2013, Journal of Improbable Astronomy, 1, 1
%\bibitem[\protect\citeauthoryear{Others}{2013}]{Others2013}
%Others S., 2012, Journal of Interesting Stuff, 17, 198
%\end{thebibliography}

%%%%%%%%%%%%%%%%%%%%%%%%%%%%%%%%%%%%%%%%%%%%%%%%%%

%%%%%%%%%%%%%%%%% APPENDICES %%%%%%%%%%%%%%%%%%%%%

\appendix

\section{Some extra material}

If you want to present additional material which would interrupt the flow of the main paper,
it can be placed in an Appendix which appears after the list of references.

%%%%%%%%%%%%%%%%%%%%%%%%%%%%%%%%%%%%%%%%%%%%%%%%%%


% Don't change these lines
\bsp	% typesetting comment
\label{lastpage}
\end{document}

% End of mnras_template.tex
