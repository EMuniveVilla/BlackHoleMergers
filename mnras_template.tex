% mnras_template.tex 
%
% LaTeX template for creating an MNRAS paper
%
% v3.3 released April 2024
% (version numbers match those of mnras.cls)
%
% Copyright (C) Royal Astronomical Society 2015
% Authors:
% Keith T. Smith (Royal Astronomical Society)

% Change log
%
% v3.3 April 2024
%   Updated \pubyear to print the current year automatically
% v3.2 July 2023
%	Updated guidance on use of amssymb package
% v3.0 May 2015
%    Renamed to match the new package name
%    Version number matches mnras.cls
%    A few minor tweaks to wording
% v1.0 September 2013
%    Beta testing only - never publicly released
%    First version: a simple (ish) template for creating an MNRAS paper

%%%%%%%%%%%%%%%%%%%%%%%%%%%%%%%%%%%%%%%%%%%%%%%%%%
% Basic setup. Most papers should leave these options alone.
\documentclass[fleqn,usenatbib]{mnras}

% MNRAS is set in Times font. If you don't have this installed (most LaTeX
% installations will be fine) or prefer the old Computer Modern fonts, comment
% out the following line
\usepackage{newtxtext,newtxmath}
% Depending on your LaTeX fonts installation, you might get better results with one of these:
%\usepackage{mathptmx}
%\usepackage{txfonts}

% Use vector fonts, so it zooms properly in on-screen viewing software
% Don't change these lines unless you know what you are doing
\usepackage[T1]{fontenc}
\usepackage{siunitx}    % Adds units
% Allow "Thomas van Noord" and "Simon de Laguarde" and alike to be sorted by "N" and "L" etc. in the bibliography.
% Write the name in the bibliography as "\VAN{Noord}{Van}{van} Noord, Thomas"
\newcommand{\SL}[1]{{\textcolor{Cerulean}{[SL: #1]}}}
\newcommand{\muB}{\mu_{\rm B}}
\newcommand{\ncor}[1]{{\textcolor{black}{#1}}}
\newcommand{\nncor}[1]{{\textcolor{black}{#1}}}
\def\muS{\mu_{\rm S}}
\DeclareRobustCommand{\VAN}[3]{#2}
\let\VANthebibliography\thebibliography
\def\thebibliography{\DeclareRobustCommand{\VAN}[3]{##3}\VANthebibliography}


%%%%% AUTHORS - PLACE YOUR OWN PACKAGES HERE %%%%%

% Only include extra packages if you really need them. Avoid using amssymb if newtxmath is enabled, as these packages can cause conflicts. newtxmatch covers the same math symbols while producing a consistent Times New Roman font. Common packages are:
\usepackage{graphicx}	% Including figure files
\usepackage{amsmath}	% Advanced maths commands

%%%%%%%%%%%%%%%%%%%%%%%%%%%%%%%%%%%%%%%%%%%%%%%%%%

%%%%% AUTHORS - PLACE YOUR OWN COMMANDS HERE %%%%%

% Please keep new commands to a minimum, and use \newcommand not \def to avoid
% overwriting existing commands. Example:
%\newcommand{\pcm}{\,cm$^{-2}$}	% per cm-squared

%%%%%%%%%%%%%%%%%%%%%%%%%%%%%%%%%%%%%%%%%%%%%%%%%%

%%%%%%%%%%%%%%%%%%% TITLE PAGE %%%%%%%%%%%%%%%%%%%

% Title of the paper, and the short title which is used in the headers.
% Keep the title short and informative.
\title[Short title, max. 45 characters]{MNRAS \LaTeXe\ template -- title goes here}

% The list of authors, and the short list which is used in the headers.
% If you need two or more lines of authors, add an extra line using \newauthor
\author[K. T. Smith et al.]{
Keith T. Smith,$^{1}$\thanks{E-mail: publications@ras.ac.uk (KTS)}
A. N. Other,$^{2}$
Third Author$^{2,3}$
and Fourth Author$^{3}$
\\
% List of institutions
$^{1}$Royal Astronomical Society, Burlington House, Piccadilly, London W1J 0BQ, UK\\
$^{2}$Department, Institution, Street Address, City Postal Code, Country\\
$^{3}$Another Department, Different Institution, Street Address, City Postal Code, Country
}

% These dates will be filled out by the publisher
\date{Accepted XXX. Received YYY; in original form ZZZ}

% Prints the current year, for the copyright statements etc. To achieve a fixed year, replace the expression with a number. 
\pubyear{\the\year{}}

% Don't change these lines
\begin{document}
\label{firstpage}
\pagerange{\pageref{firstpage}--\pageref{lastpage}}
\maketitle

% Abstract of the paper
\begin{abstract}
This is a simple template for authors to write new MNRAS papers.
The abstract should briefly describe the aims, methods, and main results of the paper.
It should be a single paragraph not more than 250 words (200 words for Letters).
No references should appear in the abstract.
\end{abstract}

% Select between one and six entries from the list of approved keywords.
% Don't make up new ones.
\begin{keywords}
keyword1 -- keyword2 -- keyword3
\end{keywords}

%%%%%%%%%%%%%%%%%%%%%%%%%%%%%%%%%%%%%%%%%%%%%%%%%%

%%%%%%%%%%%%%%%%% BODY OF PAPER %%%%%%%%%%%%%%%%%%

\section{Introduction}

The study of dark matter and scalar fields around black holes, both supermassive and primordial and in binary systems, has been approached from relativistic analyses, ultralight dark matter solitons, Bose–Einstein condensates, relativistic accretion, axion stars, superfluidity and gravitational wave effects.

\subsection{Theoretical foundations: dark matter redistribution around black holes}

The presence of a massive black hole redistributes the dark matter density profile in its vicinity. The redistribution may be determined using an approach pioneered by \citep{PhysRevLett.83.1719}: begin with a model distribution function for the dark matter and ``grow'' the black hole adiabatically, holding the adiabatic invariants of the motion constant. Building upon this framework, \citep{PhysRevD.88.063522} carries out the calculation fully relativistically using the exact Schwarzschild geometry of the black hole, finding that the dark matter density generically vanishes at $r=2R_{\rm S}$, not $4R_{\rm S}$ as in the Newtonian Gondolo--Silk approach (where $R_{\rm S}$ is the Schwarzschild radius). The spike very close to the black hole reaches significantly higher densities, with gravitational effects shown to be significantly smaller than the relativistic effects of the black hole (including frame dragging and quadrupolar effects) for stars orbiting close to the black hole that might test black hole no-hair theorems.

The mass-radius relation of self-gravitating Bose-Einstein condensates with an attractive $-1/r$ external potential created by a central mass is characterized in \citep{chavanis2019massradiusrelationselfgravitatingboseeinstein}, where an analytical Gaussian ansatz approach is used to study both noninteracting and self-interacting bosons. These results apply to dark matter halos made of self-gravitating Bose-Einstein condensates where a central mass mimics a supermassive black hole, demonstrating how central black holes affect mass-radius relations and maximum masses of axionic halos.

\subsection{Ultralight and fuzzy dark matter as black hole environments}

Measurements of the dynamical environment of supermassive black holes (SMBHs) are becoming increasingly precise through stellar velocity measurements and Event Horizon Telescope imaging. \citep{Bar_2019} searches for ultralight dark matter solitons near Sgr A* and M87*, setting constraints that exclude solitons for particle masses $2\times10^{-20}$--$8\times10^{-19}$ eV (Sgr A*) and $\lesssim4\times10^{-22}$ eV (M87*), while showing that SMBH dynamical effects can suppress soliton masses by orders of magnitude.

The effect of a supermassive black hole on the density profile of a fuzzy dark matter (FDM) soliton core at the centre of a dark matter halo reveals a `squeezing' effect \citep{10.1093/mnras/staa202}. Numerical solutions of the Schrödinger-Poisson equations demonstrate that the black hole decreases the soliton core radius while increasing central density. Applying this analysis to M87 and the Milky Way with observational constraints constrains the FDM particle mass to exclude the range $10^{-22.12}$--$10^{-22.06}$ eV.

\subsection{Scalar field accretion and axion stars}

The relativistic Bondi accretion of dark matter onto a non-spinning black hole, assuming the dominant halo component is a Standard Model gauge-singlet scalar, is self-consistently solved in \citep{Feng_2022}. The scalar mass ($m\simeq10^{-5}$ eV) and quartic self-coupling ($\lambda\lesssim10^{-19}$) are constrained to be compatible with galactic halo properties. The spike density profile is better represented by a piecewise double-power law, ranging from $\rho_0(r)\propto r^{-1.20}$ near the sound horizon to $\rho_0(r)\propto r^{-1.00}$ toward the Bondi radius.

The density profile of superfluid dark matter around supermassive black holes at galactic centres exhibits distinct power-law behaviours that distinguish superfluid from collisionless dark matter predictions \citep{De_Luca_2023}, depending on the fluid equation of state.

A general class of axion models, including QCD and string axions, yields axion star formation in virialized dark minihalos around primordial black holes through gravitational Bose-Einstein condensation \citep{PhysRevD.102.023013}. Conditions for minihalos to kinetically produce axion stars before galaxy formation are determined, expecting up to $\sim 10^{17}$ (or $\sim 10^9$ for string axions) axion stars within a 100 parsec radius around the Sun. Novel findings on the parameter space of axion stars reveal distinct morphological characteristics in dense minihalos potentially surrounding primordial black holes and in axion miniclusters \citep{Yin_2024}. Applications to bound states between primordial black holes and axion stars provide observational constraints from gravitational microlensing and microlensing events in the EROS-2 survey.

\subsection{Superradiant cloud formation in binary systems}

Superradiant instability forms clouds around rotating black holes composed of ultralight bosonic fields, such as axions. The evolution of axion clouds in binaries during inspiral, including axion self-interaction effects, reveals rich phenomena with gravitational wave signatures \citep{takahashi2024selfinteractingaxioncloudsrotating}. When self-interaction is significant, two types of clouds coexist through mode coupling, with signatures imprinted in gravitational-wave phase modifications and dynamical instability (bosenova) possible during the binary inspiral phase.

\subsection{Self-interactions and dephasing in binary black holes}

For dark matter to be detectable with gravitational waves from binary black holes, it must reach higher than average densities in their vicinity. The dephasing effect on the last ten orbits of an equal-mass binary is maximized when the Compton wavelength of the scalar particle is comparable to the orbital separation \citep{PhysRevLett.132.211401}. This phenomenology differs from dynamical friction and radiation of energy/angular momentum; instead, it is dominated by the radial force towards the overdensity between the black holes.

Gravitational waves provide crucial insights about black hole environments. Numerical relativity simulations reveal the impact of self-interacting scalar dark matter clouds on isolated and binary black holes \citep{PhysRevD.110.083011}. Repulsive self-interactions smoothen the density spike and reduce dephasing, while attractive self-interactions enhance growth but can trigger bosenova-like explosions disrupting the cloud. The impact on equal-mass black hole mergers is quantified via gravitational-wave dephasing across coupling ranges.

Resonant interactions between binary black hole systems and solitons (self-gravitating configurations of ultralight bosonic dark matter) induce metric perturbations and generate distinct oscillatory patterns in gravitational waves \citep{ybtp-fzwl}. These oscillatory patterns could be detected with future detectors such as the Laser Interferometer Space Antenna, providing evidence for solitons.

\subsection{Boson stars and exotic compact objects}

Light scalar particles are well-motivated dark matter candidates arising naturally in many Standard Model extensions. Self-gravitating condensates (boson stars) formed from scalar ultra-light dark matter (ULDM) can host central black holes \citep{banik2025bosonstarshostingblack}, with hydrostatic equilibrium solved in the non-relativistic limit consistently incorporating the black hole's gravitational potential. The inspiral of boson-star–black-hole systems is examined with gravitational-wave dephasing quantified due to the ULDM environment.

The first systematic search for exotic compact mergers in Advanced LIGO and Virgo events compares gravitational-wave signals to numerical simulations of head-on mergers of horizonless exotic compact objects (Proca stars) \citep{CalderonBustillo:2022cja}. GW190521 and GW200220 yield boson masses of $\mu_{\rm B} = 8.69^{+0.61}_{-0.75}\times10^{-13}$ eV and $\mu_{\rm B} = 9.13^{+1.18}_{-1.30}\times10^{-13}$ eV at $90\%$ credible level.

\subsection{Superradiance and black-hole spin constraints}

Ultralight bosons and axion-like particles arise naturally in several extensions of the Standard Model and produce model-independent signatures through gravity. The superradiant instability of spinning black holes extracts angular momentum to form long-lived bosonic clouds and constrains the maximum spin of astrophysical black holes. The spectrum of the most unstable modes of a massive vector (Proca) field for generic black-hole spin and Proca mass has been computed \citep{Cardoso_2018}.
Observed stability of inner accretion disks around stellar-mass black holes provides direct constraints on dark-photon masses in the range $10^{-13}\,\mathrm{eV}\lesssim m_V \lesssim 3\times10^{-12}\,\mathrm{eV}$; when higher azimuthal modes are included, similar bounds apply to axion-like particles in the range $6\times10^{-13}\,\mathrm{eV}\lesssim m_{\rm ALP} \lesssim 10^{-11}\,\mathrm{eV}$ \citep{Cardoso_2018}. Indirect bounds derived from mass and spin distributions of supermassive black holes—measured via continuum fitting, K$\\alpha$ iron-line methods, or with future LISA observations—constrain masses roughly in the range $10^{-19}\,\mathrm{eV}\lesssim m_V,m_{\rm ALP} \lesssim 10^{-13}\,\mathrm{eV}$ for both dark photons and axion-like particles \citep{Cardoso_2018}.

Overall, superradiance probes approximately eight orders of magnitude in ultralight boson mass \citep{Cardoso_2018}.


\subsection{Gravitational wave signals from scalar field mergers}

Scalar fields with masses between $10^{-21}$ and $10^{-11} \rm{eV}/c^2$ exhibit enhanced gravitational interactions with black holes, forming scalar clouds that modify coalescing binary dynamics \citep{Cheng:2025wac}. Numerical simulations of black-hole mergers with mass ratios $q=1$ and $q=1/2$ immersed in scalar field overdensities reveal that binaries can undergo delayed or accelerated mergers relative to vacuum.

Light scalar particles trigger growth of dense scalar configurations that alter binary dynamics and imprint signatures on gravitational-wave signals \citep{roy2025scalarfieldsblackhole}. A semi-analytic waveform model for binaries in scalar environments, validated against numerical relativity simulations, constrains scalar environments around compact binaries with tentative evidence found in GW190728.

Gravitational waves emitted by massive black hole binaries are affected by ultra-light, vector-field dark-matter environments that induce oscillatory gravitational potentials \citep{Chase:2025wwj}. The environmental effect on gravitational-wave phase is computed using the stationary-phase approximation within the post-Newtonian formalism, with detectability estimated for future space-borne interferometers such as LISA.

\subsection{Compact object mergers: from binary neutron stars to supermassive black holes}

Binary neutron star mergers provide a laboratory for probing fundamental physics through gravitational-wave emission. The dynamics of light dark matter (minimally coupled scalar field) surrounding binary neutron star systems reveals whether the scalar field remains bound over late inspiral-merger timescales \citep{Srikanth:2025lic}. The scalar field forms a common cloud around the binary that does not disperse in many scenarios, with measurable effects including binary inspiral dephasing, though effects remain small for astrophysically motivated densities.

Ultralight (or fuzzy) dark matter (ULDM) is an alternative to cold dark matter characterized by solitonic cores at collapsed halo centres. These cores increase drag on supermassive black hole (SMBH) binaries, changing merger dynamics and the gravitational-wave background \citep{Boey:2025qbo}. Detailed simulations of high-mass SMBH binaries within massive halo solitons find more rapid decay than previous simulations, with ULDM particle masses greater than $10^{-21}$ eV potentially alleviating the final parsec problem.

\subsection{The final parsec problem and ULDM solutions}

When two galaxies merge, they produce a supermassive black hole binary at their centre. Numerical simulations show that SMBHBs typically stall at parsec separations for billions of years, the final parsec problem. ULDM halos around SMBHBs generate dark matter waves via dynamical friction that carry away orbital energy, rapidly driving black holes together \citep{Koo:2023gfm}. Numerical simulations of black hole binaries inside ULDM halos demonstrate that gravitational cooling and quasi-normal modes avoid the loss-cone problem, with the decay timescale providing lower bounds on ULDM particle and SMBH masses consistent with observational data.



\section{Conclusions}

The last numbered section should briefly summarise what has been done, and describe
the final conclusions which the authors draw from their work.

\section*{Acknowledgements}

The Acknowledgements section is not numbered. Here you can thank helpful
colleagues, acknowledge funding agencies, telescopes and facilities used etc.
Try to keep it short.

%%%%%%%%%%%%%%%%%%%%%%%%%%%%%%%%%%%%%%%%%%%%%%%%%%
\section*{Data Availability}

 
The inclusion of a Data Availability Statement is a requirement for articles published in MNRAS. Data Availability Statements provide a standardised format for readers to understand the availability of data underlying the research results described in the article. The statement may refer to original data generated in the course of the study or to third-party data analysed in the article. The statement should describe and provide means of access, where possible, by linking to the data or providing the required accession numbers for the relevant databases or DOIs.




%%%%%%%%%%%%%%%%%%%% REFERENCES %%%%%%%%%%%%%%%%%%

% The best way to enter references is to use BibTeX:

\bibliographystyle{mnras}
\bibliography{example} % if your bibtex file is called example.bib


% Alternatively you could enter them by hand, like this:
% This method is tedious and prone to error if you have lots of references
%\begin{thebibliography}{99}
%\bibitem[\protect\citeauthoryear{Author}{2012}]{Author2012}
%Author A.~N., 2013, Journal of Improbable Astronomy, 1, 1
%\bibitem[\protect\citeauthoryear{Others}{2013}]{Others2013}
%Others S., 2012, Journal of Interesting Stuff, 17, 198
%\end{thebibliography}

%%%%%%%%%%%%%%%%%%%%%%%%%%%%%%%%%%%%%%%%%%%%%%%%%%

%%%%%%%%%%%%%%%%% APPENDICES %%%%%%%%%%%%%%%%%%%%%

\appendix

\section{Some extra material}

If you want to present additional material which would interrupt the flow of the main paper,
it can be placed in an Appendix which appears after the list of references.

%%%%%%%%%%%%%%%%%%%%%%%%%%%%%%%%%%%%%%%%%%%%%%%%%%


% Don't change these lines
\bsp	% typesetting comment
\label{lastpage}
\end{document}

% End of mnras_template.tex
