% mnras_template.tex 
%
% LaTeX template for creating an MNRAS paper
%
% v3.3 released April 2024
% (version numbers match those of mnras.cls)
%
% Copyright (C) Royal Astronomical Society 2015
% Authors:
% Keith T. Smith (Royal Astronomical Society)

% Change log
%
% v3.3 April 2024
%   Updated \pubyear to print the current year automatically
% v3.2 July 2023
%	Updated guidance on use of amssymb package
% v3.0 May 2015
%    Renamed to match the new package name
%    Version number matches mnras.cls
%    A few minor tweaks to wording
% v1.0 September 2013
%    Beta testing only - never publicly released
%    First version: a simple (ish) template for creating an MNRAS paper

%%%%%%%%%%%%%%%%%%%%%%%%%%%%%%%%%%%%%%%%%%%%%%%%%%
% Basic setup. Most papers should leave these options alone.
\documentclass[fleqn,usenatbib]{mnras}

% MNRAS is set in Times font. If you don't have this installed (most LaTeX
% installations will be fine) or prefer the old Computer Modern fonts, comment
% out the following line
\usepackage{newtxtext,newtxmath}
% Depending on your LaTeX fonts installation, you might get better results with one of these:
%\usepackage{mathptmx}
%\usepackage{txfonts}

% Use vector fonts, so it zooms properly in on-screen viewing software
% Don't change these lines unless you know what you are doing
\usepackage[T1]{fontenc}
\usepackage{siunitx}    % Adds units
% Allow "Thomas van Noord" and "Simon de Laguarde" and alike to be sorted by "N" and "L" etc. in the bibliography.
% Write the name in the bibliography as "\VAN{Noord}{Van}{van} Noord, Thomas"
\newcommand{\SL}[1]{{\textcolor{Cerulean}{[SL: #1]}}}
\newcommand{\muB}{\mu_{\rm B}}
\newcommand{\ncor}[1]{{\textcolor{black}{#1}}}
\newcommand{\nncor}[1]{{\textcolor{black}{#1}}}
\def\muS{\mu_{\rm S}}
\DeclareRobustCommand{\VAN}[3]{#2}
\let\VANthebibliography\thebibliography
\def\thebibliography{\DeclareRobustCommand{\VAN}[3]{##3}\VANthebibliography}


%%%%% AUTHORS - PLACE YOUR OWN PACKAGES HERE %%%%%

% Only include extra packages if you really need them. Avoid using amssymb if newtxmath is enabled, as these packages can cause conflicts. newtxmatch covers the same math symbols while producing a consistent Times New Roman font. Common packages are:
\usepackage{graphicx}	% Including figure files
\usepackage{amsmath}	% Advanced maths commands

%%%%%%%%%%%%%%%%%%%%%%%%%%%%%%%%%%%%%%%%%%%%%%%%%%

%%%%% AUTHORS - PLACE YOUR OWN COMMANDS HERE %%%%%

% Please keep new commands to a minimum, and use \newcommand not \def to avoid
% overwriting existing commands. Example:
%\newcommand{\pcm}{\,cm$^{-2}$}	% per cm-squared

%%%%%%%%%%%%%%%%%%%%%%%%%%%%%%%%%%%%%%%%%%%%%%%%%%

%%%%%%%%%%%%%%%%%%% TITLE PAGE %%%%%%%%%%%%%%%%%%%

% Title of the paper, and the short title which is used in the headers.
% Keep the title short and informative.
\title[Short title, max. 45 characters]{MNRAS \LaTeXe\ template -- title goes here}

% The list of authors, and the short list which is used in the headers.
% If you need two or more lines of authors, add an extra line using \newauthor
\author[K. T. Smith et al.]{
Keith T. Smith,$^{1}$\thanks{E-mail: publications@ras.ac.uk (KTS)}
A. N. Other,$^{2}$
Third Author$^{2,3}$
and Fourth Author$^{3}$
\\
% List of institutions
$^{1}$Royal Astronomical Society, Burlington House, Piccadilly, London W1J 0BQ, UK\\
$^{2}$Department, Institution, Street Address, City Postal Code, Country\\
$^{3}$Another Department, Different Institution, Street Address, City Postal Code, Country
}

% These dates will be filled out by the publisher
\date{Accepted XXX. Received YYY; in original form ZZZ}

% Prints the current year, for the copyright statements etc. To achieve a fixed year, replace the expression with a number. 
\pubyear{\the\year{}}

% Don't change these lines
\begin{document}
\label{firstpage}
\pagerange{\pageref{firstpage}--\pageref{lastpage}}
\maketitle

% Abstract of the paper
\begin{abstract}
This is a simple template for authors to write new MNRAS papers.
The abstract should briefly describe the aims, methods, and main results of the paper.
It should be a single paragraph not more than 250 words (200 words for Letters).
No references should appear in the abstract.
\end{abstract}

% Select between one and six entries from the list of approved keywords.
% Don't make up new ones.
\begin{keywords}
keyword1 -- keyword2 -- keyword3
\end{keywords}

%%%%%%%%%%%%%%%%%%%%%%%%%%%%%%%%%%%%%%%%%%%%%%%%%%

%%%%%%%%%%%%%%%%% BODY OF PAPER %%%%%%%%%%%%%%%%%%

\section{Introduction}

The study of dark matter and scalar fields around black holes, both supermassive and primordial and in binary systems, has been approached from relativistic analyses, ultralight dark matter solitons, Bose–Einstein condensates, relativistic accretion, axion stars, superfluidity and gravitational-wave effects.

\subsection{Theoretical foundations: dark matter redistribution around black holes}

The presence of a massive black hole redistributes the dark matter density profile in its vicinity; the adiabatic-growth method of \citep{PhysRevLett.83.1719} starts from a model distribution function and grows the black hole holding adiabatic invariants fixed, while a fully relativistic treatment using the Schwarzschild geometry finds the density vanishes at $r=2R_{\rm S}$ (not $4R_{\rm S}$) and produces much higher inner spikes \citep{PhysRevD.88.063522}.

The mass–radius relation of self-gravitating Bose–Einstein condensates in a $-1/r$ external potential created by a central mass was characterised using a Gaussian ansatz, showing how a central mass modifies maximum masses and radii of axionic halos \citep{chavanis2019massradiusrelationselfgravitatingboseeinstein}.

\subsection{Ultralight and fuzzy dark matter as black-hole environments}

Recent dynamical measurements and Event Horizon Telescope imaging enable searches for ultralight-soliton cores near SMBHs; stellar-velocity and EHT constraints exclude naive soliton-halo extrapolations for particle masses $2\times10^{-20}$--$8\times10^{-19}$ eV (Sgr A*) and $\lesssim4\times10^{-22}$ eV (M87*) and show SMBH dynamics can suppress soliton masses by orders of magnitude \citep{Bar_2019}.

Numerical solutions of the Schrödinger–Poisson equations demonstrate a `squeezing' of FDM soliton cores by central black holes (reduced core radius, increased central density), producing constraints on FDM masses (excluding $10^{-22.12}$--$10^{-22.06}$ eV under current assumptions) when applied to M87 and the Milky Way \citep{10.1093/mnras/staa202}.

\subsection{Scalar-field accretion and axion stars}

Relativistic Bondi accretion of a gauge-singlet scalar onto a non-spinning black hole constrains scalar masses ($m\simeq10^{-5}$ eV) and quartic couplings ($\lambda\lesssim10^{-19}$), and yields a piecewise double-power-law inner spike profile ($\rho_0\propto r^{-1.20}$ to $r^{-1.00}$) within the self-gravitating regime \citep{Feng_2022}.

The density profile of superfluid dark matter around SMBHs exhibits distinct power-law behaviours depending on the equation of state, enabling distinction from collisionless predictions \citep{De_Luca_2023}.

Axion-star formation via gravitational Bose–Einstein condensation in virialised minihalos around primordial black holes is predicted in broad classes of axion models, with potentially large local populations and microlensing constraints from surveys such as EROS-2 \citep{PhysRevD.102.023013,Yin_2024}.

\subsection{Superradiant cloud formation in binary systems}

Superradiant instability around rotating black holes populates bosonic bound states and, in binaries, leads to complex evolution when self-interactions and tidal effects are included; mode coupling can produce coexisting cloud families and dynamical instabilities (bosenova) that imprint on gravitational-wave phases \citep{takahashi2024selfinteractingaxioncloudsrotating}.

\subsection{Self-interactions and dephasing in binary black holes}

Wave-like dark-matter overdensities between binary black holes can produce dephasing distinct from dynamical friction, maximising when the scalar Compton wavelength is comparable to the orbital separation ($2\pi/\mu\sim d$) \citep{PhysRevLett.132.211401}.

Numerical-relativity studies show that self-interactions modify cloud evolution: repulsive couplings saturate densities and reduce dephasing, while attractive couplings enhance growth and can trigger bosenova-like explosions that reduce local signal imprint; the resulting dephasing across coupling ranges has been quantified \citep{PhysRevD.110.083011}.

Resonant interactions between binaries and solitons induce oscillatory gravitational-wave signatures that could be probed by future detectors such as LISA \citep{ybtp-fzwl}.

\subsection{Boson stars and exotic compact objects}

Self-gravitating condensates (boson stars) composed of ULDM can host central black holes and exhibit configurations determined by hydrostatic equilibrium including the black-hole potential; their inspirals produce characteristic dephasing that can probe ULDM mass and coupling space \citep{banik2025bosonstarshostingblack}.

Searches comparing LIGO/Virgo events to Proca-star simulations provide constraints and candidate interpretations for events such as GW190521 and GW200220, yielding boson-mass posteriors at $\sim10^{-12}$--$10^{-13}$ eV scales under the Proca hypothesis \citep{CalderonBustillo:2022cja}.

\subsection{Superradiance and black-hole spin constraints}

Analytical calculations of vector superradiant growth rates show spin-1 bound states can grow much faster than spin-0 states, enabling constraints on weakly-coupled spin-1 particle masses from rapidly spinning X-ray binaries (approximately $5\times10^{-14}$–$2\times10^{-11}$~eV) and lower-significance constraints from supermassive-BH spins at lighter masses \citep{PhysRevD.96.035019}.

Indirect and direct bounds derived from black-hole mass and spin measurements constrain dark-photon and axion-like masses over roughly $10^{-19}$--$10^{-11}$~eV, with superradiance probing roughly eight orders of magnitude in particle mass \citep{Cardoso_2018}.

\subsection{Gravitational-wave signals from scalar-field mergers}

Numerical simulations of black-hole mergers in scalar overdensities show that binaries may merge earlier or later than in vacuum, with mass-dependent impacts on gravitational and scalar radiation across mass ratios $q=1$ and $q=1/2$ \citep{Cheng:2025wac}.

Semi-analytic waveform models validated against numerical relativity enable Bayesian constraints on scalar environments in catalogues such as LIGO–Virgo–KAGRA and can yield tentative evidence for light-scalar environments in individual events \citep{roy2025scalarfieldsblackhole}.

Ultra-light vector-field environments induce oscillatory potentials that perturb binaries and imprint phase shifts computable within post-Newtonian stationary-phase approximations, with detectability forecasts for LISA-scale observations \citep{Chase:2025wwj}.

\subsection{Compact-object mergers: from binary neutron stars to supermassive black holes}

Light scalar fields minimally coupled to gravity can form lasting clouds around binary neutron stars, producing small dephasing and post-merger structure changes for sufficiently high densities, though effects remain small for astrophysical densities \citep{Srikanth:2025lic}.

ULDM solitonic cores accelerate SMBH binary decay and can alleviate the final-parsec stalling; detailed simulations demonstrate more rapid decay and provide lower bounds on ULDM particle and SMBH masses consistent with observations \citep{Boey:2025qbo,Koo:2023gfm}.

\subsection{The final parsec problem and ULDM solutions}

When galaxies merge they can form SMBHBs that stall at parsec scales; ULDM haloes can generate waves via dynamical friction and gravitational cooling that carry away orbital energy and drive rapid decay, offering a viable route past the final-parsec barrier \citep{Koo:2023gfm}.



\section{Conclusions}

The last numbered section should briefly summarise what has been done, and describe
the final conclusions which the authors draw from their work.

\section*{Acknowledgements}

The Acknowledgements section is not numbered. Here you can thank helpful
colleagues, acknowledge funding agencies, telescopes and facilities used etc.
Try to keep it short.

%%%%%%%%%%%%%%%%%%%%%%%%%%%%%%%%%%%%%%%%%%%%%%%%%%
\section*{Data Availability}

 
The inclusion of a Data Availability Statement is a requirement for articles published in MNRAS. Data Availability Statements provide a standardised format for readers to understand the availability of data underlying the research results described in the article. The statement may refer to original data generated in the course of the study or to third-party data analysed in the article. The statement should describe and provide means of access, where possible, by linking to the data or providing the required accession numbers for the relevant databases or DOIs.




%%%%%%%%%%%%%%%%%%%% REFERENCES %%%%%%%%%%%%%%%%%%

% The best way to enter references is to use BibTeX:

\bibliographystyle{mnras}
\bibliography{example} % if your bibtex file is called example.bib


% Alternatively you could enter them by hand, like this:
% This method is tedious and prone to error if you have lots of references
%\begin{thebibliography}{99}
%\bibitem[\protect\citeauthoryear{Author}{2012}]{Author2012}
%Author A.~N., 2013, Journal of Improbable Astronomy, 1, 1
%\bibitem[\protect\citeauthoryear{Others}{2013}]{Others2013}
%Others S., 2012, Journal of Interesting Stuff, 17, 198
%\end{thebibliography}

%%%%%%%%%%%%%%%%%%%%%%%%%%%%%%%%%%%%%%%%%%%%%%%%%%

%%%%%%%%%%%%%%%%% APPENDICES %%%%%%%%%%%%%%%%%%%%%

\appendix

\section{Some extra material}

If you want to present additional material which would interrupt the flow of the main paper,
it can be placed in an Appendix which appears after the list of references.

%%%%%%%%%%%%%%%%%%%%%%%%%%%%%%%%%%%%%%%%%%%%%%%%%%


% Don't change these lines
\bsp	% typesetting comment
\label{lastpage}
\end{document}

% End of mnras_template.tex
