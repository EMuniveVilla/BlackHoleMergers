% mnras_template.tex 
%
% LaTeX template for creating an MNRAS paper
%
% v3.3 released April 2024
% (version numbers match those of mnras.cls)
%
% Copyright (C) Royal Astronomical Society 2015
% Authors:
% Keith T. Smith (Royal Astronomical Society)

% Change log
%
% v3.3 April 2024
%   Updated \pubyear to print the current year automatically
% v3.2 July 2023
%	Updated guidance on use of amssymb package
% v3.0 May 2015
%    Renamed to match the new package name
%    Version number matches mnras.cls
%    A few minor tweaks to wording
% v1.0 September 2013
%    Beta testing only - never publicly released
%    First version: a simple (ish) template for creating an MNRAS paper

%%%%%%%%%%%%%%%%%%%%%%%%%%%%%%%%%%%%%%%%%%%%%%%%%%
% Basic setup. Most papers should leave these options alone.
\documentclass[fleqn,usenatbib]{mnras}

% MNRAS is set in Times font. If you don't have this installed (most LaTeX
% installations will be fine) or prefer the old Computer Modern fonts, comment
% out the following line
\usepackage{newtxtext,newtxmath}
% Depending on your LaTeX fonts installation, you might get better results with one of these:
%\usepackage{mathptmx}
%\usepackage{txfonts}

% Use vector fonts, so it zooms properly in on-screen viewing software
% Don't change these lines unless you know what you are doing
\usepackage[T1]{fontenc}
\usepackage{siunitx}    % Adds units
% Allow "Thomas van Noord" and "Simon de Laguarde" and alike to be sorted by "N" and "L" etc. in the bibliography.
% Write the name in the bibliography as "\VAN{Noord}{Van}{van} Noord, Thomas"
\DeclareRobustCommand{\VAN}[3]{#2}
\let\VANthebibliography\thebibliography
\def\thebibliography{\DeclareRobustCommand{\VAN}[3]{##3}\VANthebibliography}


%%%%% AUTHORS - PLACE YOUR OWN PACKAGES HERE %%%%%

% Only include extra packages if you really need them. Avoid using amssymb if newtxmath is enabled, as these packages can cause conflicts. newtxmatch covers the same math symbols while producing a consistent Times New Roman font. Common packages are:
\usepackage{graphicx}	% Including figure files
\usepackage{amsmath}	% Advanced maths commands

%%%%%%%%%%%%%%%%%%%%%%%%%%%%%%%%%%%%%%%%%%%%%%%%%%

%%%%% AUTHORS - PLACE YOUR OWN COMMANDS HERE %%%%%

% Please keep new commands to a minimum, and use \newcommand not \def to avoid
% overwriting existing commands. Example:
%\newcommand{\pcm}{\,cm$^{-2}$}	% per cm-squared

%%%%%%%%%%%%%%%%%%%%%%%%%%%%%%%%%%%%%%%%%%%%%%%%%%

%%%%%%%%%%%%%%%%%%% TITLE PAGE %%%%%%%%%%%%%%%%%%%

% Title of the paper, and the short title which is used in the headers.
% Keep the title short and informative.
\title[Short title, max. 45 characters]{MNRAS \LaTeXe\ template -- title goes here}

% The list of authors, and the short list which is used in the headers.
% If you need two or more lines of authors, add an extra line using \newauthor
\author[K. T. Smith et al.]{
Keith T. Smith,$^{1}$\thanks{E-mail: publications@ras.ac.uk (KTS)}
A. N. Other,$^{2}$
Third Author$^{2,3}$
and Fourth Author$^{3}$
\\
% List of institutions
$^{1}$Royal Astronomical Society, Burlington House, Piccadilly, London W1J 0BQ, UK\\
$^{2}$Department, Institution, Street Address, City Postal Code, Country\\
$^{3}$Another Department, Different Institution, Street Address, City Postal Code, Country
}

% These dates will be filled out by the publisher
\date{Accepted XXX. Received YYY; in original form ZZZ}

% Prints the current year, for the copyright statements etc. To achieve a fixed year, replace the expression with a number. 
\pubyear{\the\year{}}

% Don't change these lines
\begin{document}
\label{firstpage}
\pagerange{\pageref{firstpage}--\pageref{lastpage}}
\maketitle

% Abstract of the paper
\begin{abstract}
This is a simple template for authors to write new MNRAS papers.
The abstract should briefly describe the aims, methods, and main results of the paper.
It should be a single paragraph not more than 250 words (200 words for Letters).
No references should appear in the abstract.
\end{abstract}

% Select between one and six entries from the list of approved keywords.
% Don't make up new ones.
\begin{keywords}
keyword1 -- keyword2 -- keyword3
\end{keywords}

%%%%%%%%%%%%%%%%%%%%%%%%%%%%%%%%%%%%%%%%%%%%%%%%%%

%%%%%%%%%%%%%%%%% BODY OF PAPER %%%%%%%%%%%%%%%%%%

\section{Introduction}
The study of dark matter and scalar fields around black holes, both supermassive and primordial and in binary systems, has been approached from relativistic analyses, ultralight dark matter solitons, Bose–Einstein condensates, relativistic accretion, axion stars, superfluidity and gravitational wave effects.
Binary neutron star mergers provide a laboratory for probing fundamental physics through their gravitational- wave emission and electromagnetic counterparts. In particular, they may allow us to explore signatures of physics beyond the Standard Model in strong-gravity regimes, such as those of dark matter. \citep{PhysRevD.88.063522} investigates the dynamics of light dark matter modeled as a minimally coupled scalar field surrounding binary neutron star systems, finding that scalar fields form bound clouds with potential effects on gravitational wave signals, though these remain undetectable with current observatories at realistic densities.


Measurements of the dynamical environment of supermassive black holes (SMBHs) are becoming abundant and precise. \citep{Bar_2019} searches for ultralight dark matter solitons using stellar velocity measurements near Sgr A* and Event Horizon Telescope imaging of M87*, setting constraints that exclude solitons predicted by naive extrapolations of the soliton-halo relation for particle masses $2\times10^{-20}$--$8\times10^{-19}$ eV (Sgr A*) and $\lesssim4\times10^{-22}$ eV (M87*), while showing that SMBH dynamical effects can suppress soliton masses by orders of magnitude.
The effect of a supermassive black hole (SMBH) on the density profile of a fuzzy dark matter (FDM) soliton core at the centre of a dark matter halo has been studied in \citep{10.1093/mnras/staa202}, here the Schrödinger-Poisson equations are numerically solved and demonstrates a `squeezing' effect where the black hole decreases the soliton core radius while increasing central density. Applying this analysis to M87 and the Milky Way with observational constraints, the authors constrain the FDM particle mass to exclude the range $10^{-22.12}$--$10^{-22.06}$ eV and show that improved mass measurements and theoretical modeling can extend these constraints further.

The mass-radius relation of self-gravitating Bose-Einstein condensates with an attractive $-1/r$ external potential created by a central mass was determined in \citep{chavanis2019massradiusrelationselfgravitatingboseeinstein} where an analytical Gaussian ansatz approach has been used to study both noninteracting and self-interacting bosons. These results apply to dark matter halos made of self-gravitating Bose-Einstein condensates where a central mass mimics a supermassive black hole, demonstrating how central black holes affect mass-radius relations and maximum masses of axionic halos, with approximate analytical results compared against exact limits.

A general class of axion models, including the QCD and string axion, in which the PQ symmetry is broken before or during inflation has been considered in \citep{PhysRevD.102.023013}, here the authors discuss axion star formation in virialized dark minihalos around primordial black holes through gravitational Bose-Einstein condensation. The authors determine conditions for minihalos to kinetically produce axion stars before galaxy formation, expecting up to $\sim 10^{17}$ (or $\sim 10^9$ for string axions) axion stars within a 100 parsec radius around the Sun.


The relativistic Bondi accretion of dark matter onto a non-spinning black hole, assuming the dominant halo component is a Standard Model gauge-singlet scalar was self-consistently solved in \citep{Feng_2022}. The study constrains the scalar mass ($m\simeq10^{-5}$ eV) and quartic self-coupling ($\lambda\lesssim10^{-19}$) to be compatible with galactic halo properties. In the hydrodynamic limit the authors find a lower bound on the accretion rate, $\dot{M}_{\rm min}=96\pi G^2M^2 m^4/\lambda\hbar^3$; for $M=10^6~{\rm M}_\odot$ this gives $\dot{M}_{\rm min}\simeq1.41\times10^{-9}~{\rm M}_\odot~{\rm yr}^{-1}$, subdominant to baryonic Eddington accretion. The spike density profile $\rho_0(r)$ within the self-gravitating regime is better represented by a piecewise double-power law, with $\rho_0(r)\propto r^{-1.20}$ near the sound horizon, $\rho_0(r)\propto r^{-1.00}$ toward the Bondi radius, and $\rho_0(r)\propto r^{-1.08}$ in between; this contrasts with steeper $\rho_0(r)\propto r^{-1.75}$ profiles for Coulomb-like self-interactions.

The density profile of superfluid dark matter around supermassive black holes at galactic centres was computed in \citep{De_Luca_2023} where it was shown that, depending on the fluid equation of state, the dark matter profile exhibits distinct power-law behaviours that can distinguish superfluid from collisionless dark matter predictions.

For dark matter to be detectable with gravitational waves from binary black holes, it must reach higher than average densities in their vicinity. In the case of light (wave-like) dark matter, the density of dark matter between the binary can be significantly enhanced by accretion from the surrounding environment. In \citep{PhysRevLett.132.211401}, it was shown that the resulting dephasing effect on the last ten orbits of an equal mass binary is maximized when the Compton wavelength of the scalar particle is comparable to the orbital separation, $2\pi/\mu\sim d$. The phenomenology of the effect is different from the channels that are usually discussed, where dynamical friction (along the orbital path) and radiation of energy and angular momentum drive the dephasing, and is rather dominated by the radial force (the spacetime curvature in the radial direction) towards the overdensity between the black holes. Whilst our numerical studies limit us to scales of the same order, this effect may persist at larger separations and/or particle masses, playing a significant role in the merger history of binaries.

Novel findings concerning the parameter space of axion stars, extended object forming in dense dark matter environments through gravitational condensation were presented in \citep{Yin_2024}. The formation within the dense minihalos that potentially surround primordial black holes and in axion miniclusters was emphasised. The study investigates the relation between the radius and mass of an axion star in these dense surroundings, revealing distinct morphological characteristics compared to isolated scenarios. The implications of these results when applied to the bound state between a primordial black hole and an axion star and the gravitational microlensing from extended objects were studied, leading to insights on the observational constraints from such ``halo'' axion stars. A constraint on the fraction of the galactic population of axion stars from their contribution to the microlensing events from the EROS-2 survey was provided using the numerical resolution of the Schr\"odinger-Poisson equation.

Gravitational wave observations have significantly broadened our capacity to explore fundamental physics beyond the Standard Model, providing crucial insights into dark matter that are inaccessible through conventional methods. Here, we investigate the resonant interactions between binary black hole systems and solitons, self-gravitating configurations of ultralight bosonic dark matter, which induce metric perturbations and generate distinct oscillatory patterns in gravitational waves. Upcoming experiments such as the Laser Interferometer Space Antenna could detect the oscillatory patterns in gravitational waveforms, providing an evidence for solitons. Because the effect relies solely on gravity, it does not assume any coupling of the dark sector to Standard Model particles, highlighting the capability of future gravitational-wave surveys to probe dark matter \citep{ybtp-fzwl}.

Superradiant instability can form clouds around rotating black holes (BHs) composed of ultralight bosonic fields, such as axions. A BH with such a cloud in a binary system exhibits rich phenomena, and gravitational waves (GWs) from the BH merger provide a means to probe axions. For the first time, we study the evolution of axion clouds in a binary system during the inspiral phase, including axion self-interaction effects. When the self-interaction is significant, unlike in the negligible case, two types of clouds coexist through mode coupling. We examine the evolution of the system considering the effects of dissipation caused by both self-interaction and tidal interaction. For tidal interaction, in addition to the processes of emission to infinity and absorption by the BH, indirect emission via transitions (both resonant and off-resonant) is also considered as a second-order perturbation. Our results demonstrate that the signatures of axion self-interaction are imprinted in the modification of the GW phase. Furthermore, we find the possibility of a dynamical instability called bosenova during the binary inspiral phase \citep{takahashi2024selfinteractingaxioncloudsrotating}.



Gravitational waves can provide crucial insights about the environments in which black holes live. In this work, we use numerical relativity simulations to study the behaviour of self-interacting scalar (wave-like) dark matter clouds accreting onto isolated and binary black holes. We find that repulsive self-interactions smoothen the ``spike'' of an isolated black hole and saturate the density. Attractive self-interactions enhance the growth and result in more cuspy profiles, but can become unstable and undergo explosions akin to the superradiant bosenova that reduce the local cloud density. We quantify the impact of self-interactions on an equal-mass black hole merger by computing the dephasing of the gravitational-wave signal for a range of couplings. We find that repulsive self-interactions saturate the density of the cloud, thereby reducing the dephasing. For attractive self-interactions, the dephasing may be larger, but if these interactions dominate prior to the merger, the dark matter can undergo bosenova during the inspiral phase, disrupting the cloud and subsequently reducing the dephasing \citep{PhysRevD.110.083011}.

Light scalar particles arise naturally in many extensions of the Standard Model and are well-motivated dark matter candidates. Gravitational interactions near black holes can trigger the growth of dense scalar configurations that, if sustained during inspiral, alter binary dynamics and imprint signatures on gravitational-wave signals. Detecting such effects would provide a novel probe of fundamental physics and dark matter. Here we develop a semi-analytic waveform model for binaries in scalar environments, validated against numerical relativity simulations, and apply it in a Bayesian analysis of the LIGO–Virgo–KAGRA catalog.
Our results set physically meaningful upper bounds on scalar environments around compact binaries. When superradiance priors are included, we find tentative evidence for such an environment in GW190728 with $\ln\mathcal{B}_{\rm vac}^{\rm env} \approx 3.5$, which would correspond to the existence of a light scalar field with mass $\sim 10^{-12}\,\mathrm{eV}$ \citep{roy2025scalarfieldsblackhole}.
\citep{banik2025bosonstarshostingblack}
\citep{Chase:2025wwj}
\citep{CalderonBustillo:2022cja}
\citep{Cheng:2025wac}
\citep{Srikanth:2025lic}
\citep{Boey:2025qbo}
\citep{Koo:2023gfm}.



\section{Conclusions}

The last numbered section should briefly summarise what has been done, and describe
the final conclusions which the authors draw from their work.

\section*{Acknowledgements}

The Acknowledgements section is not numbered. Here you can thank helpful
colleagues, acknowledge funding agencies, telescopes and facilities used etc.
Try to keep it short.

%%%%%%%%%%%%%%%%%%%%%%%%%%%%%%%%%%%%%%%%%%%%%%%%%%
\section*{Data Availability}

 
The inclusion of a Data Availability Statement is a requirement for articles published in MNRAS. Data Availability Statements provide a standardised format for readers to understand the availability of data underlying the research results described in the article. The statement may refer to original data generated in the course of the study or to third-party data analysed in the article. The statement should describe and provide means of access, where possible, by linking to the data or providing the required accession numbers for the relevant databases or DOIs.




%%%%%%%%%%%%%%%%%%%% REFERENCES %%%%%%%%%%%%%%%%%%

% The best way to enter references is to use BibTeX:

\bibliographystyle{mnras}
\bibliography{example} % if your bibtex file is called example.bib


% Alternatively you could enter them by hand, like this:
% This method is tedious and prone to error if you have lots of references
%\begin{thebibliography}{99}
%\bibitem[\protect\citeauthoryear{Author}{2012}]{Author2012}
%Author A.~N., 2013, Journal of Improbable Astronomy, 1, 1
%\bibitem[\protect\citeauthoryear{Others}{2013}]{Others2013}
%Others S., 2012, Journal of Interesting Stuff, 17, 198
%\end{thebibliography}

%%%%%%%%%%%%%%%%%%%%%%%%%%%%%%%%%%%%%%%%%%%%%%%%%%

%%%%%%%%%%%%%%%%% APPENDICES %%%%%%%%%%%%%%%%%%%%%

\appendix

\section{Some extra material}

If you want to present additional material which would interrupt the flow of the main paper,
it can be placed in an Appendix which appears after the list of references.

%%%%%%%%%%%%%%%%%%%%%%%%%%%%%%%%%%%%%%%%%%%%%%%%%%


% Don't change these lines
\bsp	% typesetting comment
\label{lastpage}
\end{document}

% End of mnras_template.tex
